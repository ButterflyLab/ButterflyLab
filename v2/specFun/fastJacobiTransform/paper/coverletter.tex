\documentclass{letter}

\usepackage{amssymb}
\usepackage{epstopdf}
\usepackage[ruled]{algorithm2e}
\usepackage{float}
\usepackage{amsmath,amsthm,bm,color,epsfig,enumerate,caption}

\addtolength{\textwidth}{1.6in}
\addtolength{\oddsidemargin}{-0.8in}
\addtolength{\textheight}{1.6in}
\addtolength{\topmargin}{-0.8in}

\name{Haizhao Yang}
\address{
Department of Mathematics\\
National University of Singapore\\
Level 4, Block S17\\
10 Lower Kent Ridge Road\\
Singapore 119076}
\signature{Haizhao Yang}
\date{\today}

\begin{document}

%------------------------------
\begin{letter}{
Editorial Board\\
Applied and Computational Harmonic Analysis
}

\opening{Dear members of the Editorial Board:}

I would like to submit the manuscript entitled
``Fast Algorithms for the Multi-dimensional Jacobi Polynomial Transform''
by James Bremer, Qiyuan Pang, and myself to Applied and Computational Harmonic Analysis.


We use the well-known observation that the solutions of Jacobi's differential 
equation can be represented via non-oscillatory phase and amplitude functions to develop 
 a fast algorithm for computing multi-dimensional Jacobi polynomial transforms.
More explicitly,  it  follows from this observation that the matrix corresponding to the 
discrete Jacobi transform is the Hadamard product of a 
numerically low-rank matrix and a multi-dimensional discrete Fourier transform (DFT) matrix.
The application of the Hadamard product can 
be carried out via $O(1)$ fast Fourier transforms (FFTs), resulting in a nearly optimal algorithm to compute the multidimensional Jacobi polynomial transform. 

Thank you very much for your attention and please let me know if
there is anything else I can help with.

\closing{With best regards,}

\end{letter}
\end{document}
